\documentclass{article}
\newcommand\book{Advanced Programming in the Unix Environment 3e}
\newcommand\class{CIS 5950}
\author{Tyler Ryan}
\usepackage[margin=2cm]{geometry}
\usepackage{parskip}
\usepackage{listings}
\usepackage{hyperref}

\renewcommand{\b}[1]{\textbf{#1}}
\begin{document}
\title{Process Relationship Notes\\From \book{} Chapter 9}
\maketitle
\tableofcontents
\newpage



\section{Process Relationsihps Summary}
In the previous chapter there are relationships between processes. First, every 
process has a parent process (the initial kernel-level process is usually its 
own parent). The parent is notified when the child terminates, and the perent can 
obtain the child's exit status. 

The previous chapter also mentioned process groups when it described the waitpid 
function and how we can wait for any process in a process group to terminate.

This chapter describes the relationships between groups of processes: sessions, 
which are made up of process groups. Job control is a feature supported by most 
UNIX systems today, and we've described how it's implemented by a shell that 
supports job control. The controlling terminal for a process, /dev/tty, isalso 
involved in these process relationsips.

We've mad enumerous references to the signals that are used in all these process 
relationships. The next chapter continues the discussion of signals, looking at 
all the UNIX System signals in detail.

\section{Terminal Logins}
\subsection{BSD Terminal Logins}
\begin{itemize}
    \item System Admin creates a file, usually /etc/ttys, that has one line per temrinal devices.
        \begin{itemize}
            \item Each line specifies the name of the devices and otehr parametes that are passed to the getty program.
            \item One parameter is the "baud rate" of the terminal.
        \end{itemize}

    \item When the system is bootstrapped, the kernel creates the Process ID 1, the init 
        process.
        \begin{itemize}
            \item This init process brings the system up multiuser.
            \item This init process reads the file /etc/ttys
            \item For every terminal deivces that allows a login, does a fork() 
                followed by an exec() of the program getty.
            \item It execs the getty program with an empty environment.
        \end{itemize}
    \item All processes at this point have superuser privileges.
        \begin{itemize}
            \item I.e. UID == 0 and EUID == 0
        \end{itemize}
    \item It is getty that calls open() for the terminal devices.
        \begin{itemize}
            \item The terminal is opened for reading and writing.
            \item If the device is a modem, the open() may delay inside the device driver until the modem is dialed and the call is answered.
            \item Once the device is opened, the device descriptors 0, 1, and 2 are set to the devices.
        \end{itemize}

    \item getty output something like "login:" and waits for the user to enter their username.
    \item When a username is entered, getty's job is complet, and it the ninvokes the "login" program similar to:
        \begin{itemize}
            \item execle("/bin/login", "login", "-p", username, (char*) 0, envp); \newline
        \end{itemize}
\end{itemize}
continue whenever. Doesn't seem super important

\subsection{Mac OS X Terminal Logins}

\subsection{Linux Terminal Logins}

\subsection{Solaris Terminal Logins}

\section{Process Groups}
\subsection{Function Prototypes}

\begin{lstlisting}[
    frame = single,
    caption = {Process Group Function Prototypes},
    label = {pgidfunctions}
    ]
    #include <unistd.h>

    /* If pid == 0, the PID of the calling process is reutrned.
     * Thus getpgid(0) == getpgrp();
     */
    pid_t getgrp(void);
    pid_t getpgid(pid_t pid);

    /* A process joins an existing process or creates a new process 
     * group by calling setgpid().
     * 
     * Sets the PID to pgid in the process whose PID equals "pid".
     * If the two arguments are equal, the process specified 
     * by "pid" becomes a process group leader.
     */
    int setpgid(pid_t pid, pid_t pgid);
\end{lstlisting}
\subsection{About Process Group}
\begin{itemize}
    \item A process group is a collection of one or more processses, usually associated with the same job that can receive signals from the same terminal.
    \item Each process group has a unique process group ID.
    \item Process Groups can have a process group leader.
        \begin{itemize}
            \item The leader is identified by its process group ID (PGID) being equal to its process ID (PID).
        \end{itemize}
    \item It is possible for a process group leader to create a process group, create processes in the group, and then terminate them.
    \item The process group still exists as long as at least one process is in the group, regardless of whether the group leader terminates.
    \item Process Group Lifetime: The period of time that begins when the group is created and ends when the last remaining process leaves the group.
        \begin{itemize}
            \item The last remaining process in the process group can either terminate or enter some other process group.
        \end{itemize}
    \item A process joins an existing process or creates a new process gropu by calling setgpid().
    \item A process can set the process group ID PGID of only itself or any of its children.
    \item A process can't change the process group ID PGID of one of its children after that child has called on of the exec() functions.
\end{itemize}

\section{Sessions}
\subsection{Function Prototypes}
\begin{lstlisting}[
frame   = single,
caption = {Session Function Prototypes},
label   = {sessionfunctions}
]
    #include <unistd.h>
        pid_t setsid(void);
            Returns: process group ID if OK, -1 on error
        pidt getsid(pid_t pid);
            Returns: session leader's process group ID PGID if OK, -1 on error.
\end{lstlisting}
% Get figure 9.6 in here
\begin{itemize}
\item A session is a collection of one or more process groups.
\item setsid() returns an error if the caller is already a Process Group Leader.
\item It is impossible for a child's PID to equal its inherited PGID.
\item getsid(0); returns the process group ID PGID of the calling process's session leader.
\item If the calling process of setsid() is not a process group leader, this function creates a new session and the following 3 things happen:
    \begin{enumerate}
        \item The process becomes the "session leader" of this new session.
            \begin{itemize}
                \item The process is the only process in this new session
            \end{itemize}
        \item The process becomes the process group leader of a new process group.
            \begin{itemize}
                \item The new process group ID PGID is the process ID of the calling process.
            \end{itemize}
        \item The process has no controlling terminal.
            \begin{itemize}
                \item If the process had a controlling terminal before calling setsid(), that association is broken.
            \end{itemize}
    \end{enumerate}
\end{itemize}

\section{Controlling Terminal}
GET Figure 9.7 HERE \newline
\begin{itemize}
\item Controlling Terminal: Usually the terminal device (in the case of aterminal login) or pseudo-terminal device (in the case of a network login) on which we log in.
\item A session can have 1 Controlling Terminal.
\item Controlling Process: A Session Leader that establishes the connection to the controlling terminal.
\item The process groups within a session can be divided into:
    \begin{itemize}
        \item 1 Foreground Process Group
        \item $\geq$ 1 Background Process Group(s)
        \item If a session has a Controlling Terminal,it has a single foreground process group, and all other process gropus in the session are background process groups.
    \end{itemize}
\item Whenever the terminal's interrupt key (DELETE or Control-C) is pressed, this causes the interrupt signal to be sent to all process in the FOREGROUND Process Group.
\item If a modem (or network) disconnect and is detected by the terminal interface, the hang-up signal (SIGHUP) is sent to the Controlling Process (ie The Session Leader).
\item Usually, we don't have to worry about the Controlling Terminal; it is established automatically when we login.
\item open() the file /dev/tty guarantees that a program is talking to the Controlling Terminal.
    \begin{itemize}
        \item Thss special file is a synonym within the kernel for the Contorlling Temrinal.
        \item If the program doesn't have a Controlling Terminal, the open() of this device will fail.
    \end{itemize}

\end{itemize}
\section{Foreground Process Group Functions}
\begin{lstlisting}[
    frame = single,
    caption = {Foreground Process Group Function Prototypes},
    label = {foregroundfunctions}
    ]
    #include <unistd.h>

    // filedes == file descriptor
    pid_t tcgetpgrp(int filedes);
        Returns: PGID of foreground process group if OK, -1 on error
    int tcsetpgrp(int fildes, pid_t pgrpid);
        Returns: 0 if OK, -1 on error
    pid_t tcgetsid(int filedes);
        Returns: Session Leader's PGID if OK, -1 on error.
\end{lstlisting}
\begin{itemize}
\item These functions tell the kernel which process group is in the foreground process group, so that the terminal device driver knows where to send the terminal input and the terminal-generated signals.
\item tcgetpgrp() returns the PGID of the foreground process group associated with the termianl open on "filedes".
\item If the process has a Controlling Terminal, the process can call tcsetpgrp() to set the foreground PGID to pgrpid.
    \begin{itemize}
        \item The value of pgrpid must be the process group ID PGID of a process group in the SAME SESSION, and filedes must refer to the Controlling Temrinal of the session.
    \end{itemize}
\item Most applications don't call these two functions directly.
    \begin{itemize}
        \item They are normally called by Job-Control Shells.
    \end{itemize}
\item Applications that need to manage Controlling Terminals can use tcgetsid() to identify the Session ID of the Controlling Terminal's Session Leader
    \begin{itemize}
        \item Controlling Terminal's Session Leader == Session Leader's PGID.
    \end{itemize}
\end{itemize}
\section{Job Control}
\begin{itemize}
\item Job: Groups of Processes.
\item Job Control: Allows the starting of multiple Jobs from a single terminal and to control which jobs can access the temrinal and which jobs are to run in the background.
\item Job Control requires 3 forms of support:
    \begin{itemize}
        \item A SHELL that supports Job Control
        \item The TERMINAL DRIVER in the kernel must support job control
        \item The KERNEL must support certain job control signals.
    \end{itemize}
\item The shell doesn't print the changed status of background jobs at any random time--only right before it prints its prompt, to let us enter a new command line.
    \begin{itemize}
        \item If the shell didn't do this, it could output while we were entering an input line.
    \end{itemize}
\item Terminal Driver looks for 3 special characters, which generate signals to the FOREGROUND Process Group (ONLY)
    \begin{enumerate}
        \item DELETE or Control C: Generates SIGINT
        \item Control-\: Generates SIGQUIT
        \item Control-Z: Generates SIGTSTP.
    \end{enumerate}
\item Only FOREGROUND jobs receive temrinal input
    \begin{itemize}
        \item It is not an error for a background job to try to read from the terminal, but the Terminal Driver detects this and sends a special signal to the background job; SIGTTIN.
            \begin{itemize}
                \item This normally stops the background process
                \item Notifies the shell (i.e. us) an we can bring it to the foreground.
            \end{itemize}
    \end{itemize}
\end{itemize}
\subsection{Job Control Examples}
\begin{lstlisting}[
    frame   = single,
    caption = {Job Control Examples},
    label   = {jobcontrolexamples}
]
    # Foreground Process
    $ ./a.out

    # Background Process
    $ ./a.out &

    # Switching between background and foreground processes
    $ cat > temp.txt &              # Start in the background
    [1]   1234
    $                               # Press Return 
    [1] + Stopped   cat > temp.foo  
    $ fg 1                          # Move to the foreground
    hello, world                    # Type in "hello, world"
    ^D                              # End of File (EOF) character
    $ cat temp.txt                  # Observe the output
    hello,world
\end{lstlisting}
\begin{itemize}
    \item Starts \textbf{cat} process in the background, 
    \item But when \textbf{cat} tries to read its standard input (the Contolling 
        Terminal), the Terminal Driver, knowing that it is a background job, sends
        the SIGTTIN signal to the background job.
    \item The shell detects the change in status of its child (\textbf{wait()}
        and \textbf{waitpid()}) and tells us that the job has been stopped.
    \item Move the job into the foreground with the shell's \textbf{fg} command.
            \begin{itemize}
                \item Doing so uses the \textbf{tcsetpgrp()} function to place the
                    job into the foreground process group.
            \end{itemize}
    \item The shell sends the signal SIGCONT to the process group.
    \item Since the job is now in the foreground process group, it can read from 
        the Controlling Terminal.\newline
\end{itemize}
\textbf{Note:} If the background job outputs to the controlling terminal, you can 
disable it with the \textbf{stty tostop} command.


\section{Shell Excecution Of Programs}
\subsection{Example with the Bourne Shell}
Command:\newline
\begin{verbatim}
$ ps -o pid,ppid,pgrp,session,tpgid,comm | cat1
        # Look at their PPIDs and PIDs.
        # This shows that "ps" and "cat1" are children of "cat2"
        # and that "cat2" is a child of the shell.
\end{verbatim}

\begin{table}[h!]
    \begin{center}
        \begin{tabular}{l l l l l l}
            PID      & PPID & PGID     & SID & COMMAND \\
            \hline
            2837     & 2818 & 2837     & 2837 & 5799   \\
            \b{5799} & 2837 & \b{5799} & 2837 & 5799   \\
            \b{5800} & 2837 & \b{5799} & 2837 & 5799   \\
        \end{tabular}
    \end{center}
\end{table}

\subsection{Foreground Example With The Bourne Again Shell (BASH)}
Command:
\begin{verbatim}
$ ps -o pid,ppid,pgrp,session,tpgid,comm | cat1
\end{verbatim}
\begin{table}[h!]
    \begin{center}
        \begin{tabular}{l l l l l l}
            PID      & PPID & PGRP     & SESS & TPGID & COMMAND \\
            \hline
            2837     & 2818 & 2837     & 2837 & 5799  & bash    \\
            \b{5799} & 2837 & \b{5799} & 2837 & 5799  & ps      \\
            \b{5800} & 2837 & \b{5799} & 2837 & 5799  & cat1    \\
        \end{tabular}
    \end{center}
\end{table}
\begin{itemize}
    \item This example starts the jobs in the foreground.
    \item Foreground processes are in bold.
    \item Both processes, "ps" and "cat1", are placed into the new process group
        (PGRP) 5799 (ie. the Foreground group)
    \item "ps" was also created first instead of what would've been last if 
        cat2 wasn't there for the Bourne Shell.
\end{itemize}

\subsection{Background Example With The Bourne Again Shell (BASH)}
\large{This example starts the jobs in the background.}\newline
Command:
\begin{verbatim}
$ ps -o pid,ppid,pgrp,session,tpgid,comm | cat1 &
\end{verbatim}
\begin{table}[h!]
    \begin{center}
        \begin{tabular}{l l l l l l}
            PID      & PPID & PGRP     & SESS & TPGID & COMMAND \\
            \hline
            \b{2837} & 2818 & \b{2837} & 2837 & 2837  & bash    \\
            5801     & 2837 & 5801     & 2837 & 2837  & ps      \\
            5802     & 2837 & 5801     & 2837 & 2837  & cat1    \\
        \end{tabular}
    \end{center}
\end{table}

\begin{itemize}
    \item This example starts the jobs in the background.
    \item Foreground processes are in bold.
    \item Both processes, "ps" and "cat1", are placed into the BACKGROUND.
    \item "ps" was also created first instead of what would've been last if 
        cat2 wasn't there for the Bourne Shell.
\end{itemize}

\section{Orphaned Process Groups}

\begin{itemize}
    \item A PROCESS whose parent terminates is called an orphan and is inherited by the init process.
    \item The POSIX.1 definition of an Orphaned Process Group is one in which the parent of every member is either itself a member of the group or is not a member of the group's session.
        \begin{itemize}
            \item In other words, the process group is not orphaned as long as a process in the group has a parent in a different process group but in the same session.
        \end{itemize}
    \item POSIX.1 requires that every process in the newly orphaned process group that is stopped to be sent the hang-up signal (SIGHUP) followed by the continue signal (SIGCONT).
\end{itemize}

\section{Example Code}

\begin{lstlisting}[
    frame = single,
    caption = {Orphaned Process Example},
    language = C
]
#include <signal.h>
#include <stdio.h>
#include <errno.h>
#include <stdlib.h>
#include <unistd.h>

/* This program creates an orphaned process by ending 
 * a parent process while it's child is stopped.
 */

// What to do if SIGHUP is sent
static void sig_hup(int signo)
{
    printf("SIGHUP received, pid = %d\n", getpid());
}

// Simple debug
static void pr_ids(char * name)
{
    printf("%s: pid = %d, ppid = %d, pgrp = %d, tpgrp = %d\n",
            name, getpid(), getppid(), 
            getpgrp(), tcgetpgrp(STDIN_FILENO));
    fflush(stdout);
}

int main()
{
    char  c;
    pid_t pid;

    // First to print
    pr_ids("parent");

    pid = fork();

    if (pid < 0)
    {
        perror("Error");
        exit(EXIT_FAILURE);
    }

    // Parent
    else if (pid > 0) 
    {
        sleep(2);
        printf("Parent process ended\n");
        exit(EXIT_SUCCESS);
    }

    // Child 
    else 
    {

        // Second to print
        pr_ids("child");

        /* POSIX.1 requires that every process in the newly orphaned
         * process group that is STOPPED (as our child is) be sent 
         * the hang-up signal (SIGHUP) FOLLOWED BY THE CONTINUE
         * SIGNAL (SIGCONT).
         *
         * The default action of SIGHUP is to terminate the process.
         * This would terminate our program.
         *
         * signal() tells the program what to do in with a 
         * particular signal. Essentially allowing us to override 
         * the default action of any given signal.
         *
         * To override the default action of SIGHUP, we pass our 
         * sig_hup() function, which does not terminate the process, 
         * to signal(). Thus allowing it to continue.
         * 
         * NOTE: Without signal() or kill(), the child would still 
         * continue in the background even after the parent ended.
         *
         * In short, the kill() function is being ran and thus
         * our child is stopped. However, it becomes an orphan
         * after the parent process ends. Every stopped orphan 
         * must be continued. Thus is why our child continues.
         */

        // signal() looks to be a higher order function.
        signal(SIGHUP, sig_hup);
        printf("After signal()\n");

        printf("Program Stopped\n");
        kill(getpid(), SIGTSTP);
        printf("Continue after kill() (Being Stopped)\n");


        /* I believe this is trying to illustrate the fact that because
         * the this process is orphaned, it is sent to the background.
         * However, a background process cannot take input from the 
         * controlling terminal so it errors.
         *
         * To get this to work, you need to enter a single letter 
         * before the parent exits.
         */

        /* ssize_t status = read(STDIN_FILENO, &c, 1); */

        /* if (status != 1) */
        /*     printf("read error from controlling TTY, errno = %d\n", 
                                                     *     errno); */


        /* This demonstrates how the process is run in the background 
         * even after the parent ends.
         */

        printf("Continuing in the background.\n");
        sleep(3);
        pr_ids("\nchild");
        printf("Child process ends\n");
        printf(
        "Notice how the Parent PID (PPID) changes to 1 (ie. init)\n"
        );
        exit(EXIT_SUCCESS);
    }
}
\end{lstlisting}
\begin{lstlisting}

Output:

parent: pid = 366545, ppid = 366503, pgrp = 366545, tpgrp = 366545
child: pid = 366546, ppid = 366545, pgrp = 366545, tpgrp = 366545
After signal()
Program Stopped
Parent process ended
SIGHUP received, pid = 366546
Continue after kill() (Being Stopped)
Continuing in the background.

$
child: pid = 366546, ppid = 1, pgrp = 366545, tpgrp = 366503
Child process ends
Notice how the Parent PID (PPID) changes to 1 (ie. init)
\end{lstlisting}
\section{FreeBSD Implementation}
\subsection{Nothing Here}

\section{Exercises}
\subsection{Exercise 1}
1. Refer back to our discussion of the utmp and wtmp files. Why are the logout 
records written by the init process? \newline 
Is this handled the same way for a newtwork login?
\subsection{Exercise 2}
2. Write a small program that calls fork() and has the child create
a new session. Verify that the child becomes a process group leader and that the 
child no longer has a controlling terminal.
\end{document}
