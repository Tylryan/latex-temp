\documentclass{article}
\author{Tyler Ryan}
\newcommand\bookname{Advanced Programming in the Unix Environment 3e}
\newcommand\articletitle{ CH 8: Process Control Notes }
\newcommand\datestarted{February 2, 2023}
\newcommand\classname{APUE - CIT 5590}
\newcommand\be[1]{\textbf{\emph{#1}}}
\title{\articletitle\\\bookname\\\footnote{Fall 2023} CIT-5950}
\date{\datestarted}

\usepackage[margin=2cm]{geometry}
\usepackage{fancyhdr}
\usepackage{listings}
\usepackage{hyperref}
% \usepackage{indentfirst}
\setlength\parindent{0pt}   % \noindent by default

\begin{document}
\pagestyle{fancy}
% \fancyhead[ROE]{\rightmark} % Set to current Chapter
\fancyhead[LOE]{\articletitle} % Set to current Chapter
\fancyfoot{}
\fancyfoot[LEO]{\classname}
\fancyfoot[REO]{\thepage}
\fancyfoot[CEO]{\datestarted}
\maketitle
\tableofcontents
\newpage


\section{Chapter Summary}
A thorough understanding of the UNIX System's process control is 
essential for advanced programming. There are only a few functions
to master: \be{fork()}, the \be{exec()} family, \be{\_exit}, \be{wait()},
\be{waitpid()}.\newline
These primatives are used in many applications. The \be{fork()}
function also gave us an opportunity to look at \textbf{race conditions}.\newline

Our examination of the \be{system()} function and \textbf{process accounting}
gave us another look at all these process control functions. We also looked at another
variation of the \be{exec()} functions: \textbf{interpreter files} and how they operate.\newline

An understanding of the various User IDs and Group IDs that are provided--
Real, Effective, and Saved--is critial to writing safe set-userID programs.


\section{Process Identifiers}
% \sectionmark{Process Identifiers}
\subsection{The PID}
\begin{itemize}
    \item Process ID (PID).
    \item Every process has one.
    \item Always unique.
    \item All PIDs are non-negative numbers.
    \item Are reused after processes are terminated.
        \begin{itemize}
            \item However, using algorithms to delay this 
                reuse prevents new PIDs from being mistaken 
                for previous PIDs with the same number.
        \end{itemize}
    \item Because the PID is the only well known identifier 
        of a process that is always unique, it is often used 
        as a piece of other identifiers to gurantee uniqueness.

\end{itemize}
% Every process has a unique process ID (\textbf{PID}), which is a 
% \emph{non-negative integer}. Because the PID is the only well-known
% identifier of a process \emph{that is always unique}, it is
% often used as a piece of other identifiers, to guarantee
% uniqueness. \\

% \textbf{PIDs are also reused} after they are terminated. However, 
% using a algorithms to delay this reuse prevents new PIDs being
% mistaken for previous PIDs with the same number. \\

\subsection{Scheduler and Init Processes}

\textbf{PID 0} is usually the \emph{scheduler process} and is 
often known as the \emph{swapper}.

\textbf{PID 1} is usually the \emph{init process} and is invoked
by the kernel at the end of the bootstrap procedure.
\begin{itemize}
    \item Responsible for bringin up the UNIX System after 
        the kernel has been bootstrapped.
    \item Reads the system-dependent inizialization files 
        --\emph{/etc/rc*} or \emph{/etc/inittab} and 
        \emph{/etc/init.d/}-- and brings the system to a 
        certain state, such as multiuser.
    \item \textbf{Never Dies}.
    \item It is a normal user process, not a system process 
        within the kernel, like the swapper.
    \item Does have Superuser Privileges.
    \item Becomes the parent process of any orphaned child process.
    \item New program file is \emph{sbin/init}.
    \item Old program file was \emph{/etc/init}. \\
\end{itemize}


\subsection{Additional Process Identifiers}
In addition to the PID, there are other identifiers for every 
process (though not unique!). Below are the functions that 
return those identifiers.\newline

\newpage
\begin{lstlisting}[
    caption = {Additional Process Identifier Functions.},
    frame = single
    ]
    #include <unistd.h>

    pid_t getpid(void);     # Returns: PID of calling process.
    pid_t getppid(void);    # Returns: Parent Process ID (PPID) of calling process.
    uid_t getuid(void);     # Returns: Real User ID (UID) of calling process.
    uid_t geteuid(void);    # Returns: Effective User ID (EUID) of calling process.
    gid_t getgid(void);     # Returns: Real Group ID (GID) of calling process.
    gid_t getegid(void);    # Returns: Effective Group ID (EGID) of calling process.
\end{lstlisting}

\section{The \be{fork()} Function}
\begin{lstlisting}[
    frame   = single,
    label   = fork-func,
    caption = {fork() Prototype}
]
    #include <unistd.h>

    pid_t fork(void);       # Returns: 0 in child, PPID, or -1 if error.
\end{lstlisting}

\subsection{About \be{fork()}}
\begin{itemize}
    \item The new process created by fork() is called the \textbf{Child Process}.
    \item fork() is called once, but returned twice.
    \newline
    \item A Parent can have more than one child, so to uniquely identify it's children, the children return their PIDs.
        \begin{itemize} 
            \item{There is no function that tells the parent the PIDs of it's children.}
        \end{itemize}
    \item fork() returns 0 to a child, because a child can only have one parent.
        \begin{itemize} 
            \item{Additionally, a child can always call \emph{getppid()} to obtain the PID of the parent.}
        \end{itemize}

    \item The child process \textbf{copies} all data from the parent process.
        \begin{itemize} 
            \item{This includes variables from your source code!}
            \item{The child copies these variables, but does not alter the original values of the parent.} \newline
        \end{itemize}
\end{itemize}

\textbf{The two uses for fork are:}
\begin{enumerate}
    \item \textbf{When a process wants to duplicate itself so that the parent 
        and child can each execute different sections of code at the same time}
        \emph{This is common for network servers}--the parent waits for
        a serice request from the client. When the request arrives,
        the parent calls \emph{fork()} and lets the child handle the request.
        The parent goes back to waiting for the next service request to arrive.
    \item \textbf{When a process wants to execute a different program.}
        \emph{Common for shells.} In this case, the child does an \emph{exec()}
        right after it returns from \emph{fork()}.
\end{enumerate}

\textbf{The two main reasons for \emph{fork()} to fail are }
\begin{enumerate}
    \item If too may processes are already in the system.
        This usually means that something else is wrong.
    \item If the total number of processes for the 
        real user ID (UID) exceeds the system's limit.
        \textbf{CHILD\_MAX} specifies this maximum number.
\end{enumerate}

        Below is an example of the fork function. As a result of the sleep()
        function, the child should execute first. However, it \textbf{is normally never known
        whether the Parent Process will execute before the Child Process or vice versa.} \newline

\subsection{\be{fork()} Example Code}
% For example in C
\lstinputlisting[
    language = C, 
    frame    = single,
    caption  = {Example of the fork() function.}
]{./assets/main.c}

\newpage
    If you run: \textbf{gcc main.c \&\& ./a.out}, then the output will be:
\begin{lstlisting}[
    caption = {fork() function output to terminal.}, 
    frame   = single
]
    a write to stdout
    Before Fork
    pid = 145800, glob = 7, var = 89
    pid = 145794, glob = 6, var = 88 
\end{lstlisting}
    \hfill

    If you run: \textbf{gcc main.c \&\& ./a.out \textgreater{} temp.txt \&\& cat temp.txt}
    or run it in Vim, the output will be:

\begin{lstlisting}[
    caption = {Fork function output to file.},
    frame = single
]
    a write to stdout
    Before Fork
    pid = 146403, glob = 7, var = 89
    Before Fork
    pid = 146402, glob = 6, var = 88
\end{lstlisting}
    \hfill


\textbf{Don't think too hard about the second ''Before Fork''. 
It has to do with the way C flushes it's print statements.
Just be aware of it.}

% \newpage
\subsection{File Sharing}
One characteristic of fork is that all file descriptors that are
open in the parent are duplicated in the child.

There are two normal cases for handling the file descriptors
after a fork.
\begin{enumerate}
    \item \textbf{The parent waits for the child to complete.}
        \begin{itemize}
            \item Parent does not need to do anything with descriptors.
            \item When the child terminates, any of the shared 
                descriptors tat the child read from or wrote 
                to will ahve their file offsets updated 
                accordingly.
        \end{itemize}
    \item \textbf{Both the parent and the child go their own ways.}
        \begin{itemize}
            \item After the \emph{fork()}, the parent closes the file
                descriptors that it doesn't need, and the child does
                the same things.
            \item Neither interferes with the other's open descriptors.
                This scenario is often the case with network servers.
        \end{itemize}
\end{enumerate}

\section{Exit Functions}
\subsection{Normal Terminations}
A process can terminate normally in the following 5 ways.
\begin{itemize}
    \item \be{return:} from a main function is equivalent to \be{exit}.
    \item \be{exit():} Defined by ISO C and includes the calling of all 
        exit handlers that have been registerd by the calling \be{atexit}.
    \item \be{\_exit} and \be{\_Exit:} 
        \begin{itemize}
            \item \_Exit and \_exit are synonomous and do not flush standar I/O streams.
            \item ISO C defines \_Exit to provide a way for a process to 
                terminate without running exit handlers or signal handlers.
            \item The \_exit function is called by \be{exit()} and handles the UNIX 
                system-specific details.
        \end{itemize}
    \item \be{pthread\_exit():} calling this function from the last thread in the process.
\end{itemize}
\subsection{Abnormal Terminations}
The 3 forms of abnormal terminations are:
\begin{itemize}
    \item \be{abort():} A special case of the ntext item, as it
        generates the \textbf{SIGABRT} signal.
    \item By receiving certain \textbf{signals}.
    \item The last thread reponds to a cencellation request.
        \begin{itemize}
            \item By default, cancellation occurs in a deferred manner:
                one thread requests that another be canceled, and somtime later,
                the target thread terminates. \newline
        \end{itemize}
\end{itemize}

Additional Info Regarding Exit Functions
\begin{itemize}
    \item Regardless of how a process terminates, the same code in the kernel is eventually executed.
    \item In the case of an abnormal termination, the kernel generates the termination status
        to indicate the reason for the abnormal termination, not the process itself.
    \item Regardless of how a process was terminated, the parent of the process can obtain the termination status
        from either the \be{wait()} or the \be{waitpid()} function.
    \item If the parent terminates before the child, the \textbf{init process} becomes
        the parent process of said child process.
        \begin{itemize}
            \item This way, we're guaranteed that every process has a parent.
        \end{itemize}
    \item A process that has terminated, but whose parent has not yet \emph{waited} for it is called a \textbf{Zombie Process}.
        \begin{itemize}
            \item In other words, if a child terminates before the parent, but the parent does not call a \be{wait()} function, 
                then it is a Zombie. This is because the \be{wait()} function literally removes
                zombie processes.
            \item Ths \be{ps} command prints the state of a zombie process as \textbf{Z}.
        \end{itemize}
\end{itemize}

\section{\be{wait()} and \be{waitpid()} Functions.}
\subsection{Function Prototypes}
\begin{lstlisting}[
    frame   = single,
    % caption = {Wait function prototypes},
    label   = waitprototypes,
    caption = {Wait Function Prototypes}
    ]
    #include <sys/wait.h>
    // Also <wait.h> works on Fedora as of Feb. 2, 2023

    pid_t wait(int *statloc);
    pid_t waitpid(pid_t pid, int *statloc, int options);
\end{lstlisting}

\subsection{Wait Function Return Values}
\begin{center}
    \begin{table}[h!]
        \begin{tabular}{|l|l|}
            \hline
            Return Value of waitpid() & Description \\
            \hline
            pid == -1             & Waits for any child process. In this respect,
                                 \emph{waitpid()} is equivalent to \emph{wait()}. \\
            \hline 
            pid \textgreater{} 0  & Waits for thechild whose PID equals \emph{pid}. \\
            \hline
            pid == 0              & Waits for any child whose PID equals that of the calling process. \\
            \hline
            pid \textless         & Waits for any child whose PID equals the absolute value of \emph{pid}. \\
            \hline
        \end{tabular}
        \caption{\emph{waitpid()} return value descriptions.}
        \label{waitpidreturnvalues}
    \end{table}
\end{center}

\textbf{NOTE:} The \be{waitpid()} function returns the PID of the child that terminated and stores the child's 
termination status in the memory location pointed to by \be{statloc}.\newline

With \be{wait()}, the only real error is if the calling process has no children.
With \be{waitpid()}, it is also possible to get an error if the specified process or 
process group does not exist or is not a child of the calling process.\newline

\subsection{Useful Macros}
\be{Remember:} You can run \be{man 2 wait} and find a list of these macros.
\begin{center}
\begin{tabular}{|l | l|}
    \hline
    \textbf{Macro} & \textbf{Description} \\ %[0.5ex] 
    \hline
    WIFEXITED(status)   & True if status was returned for a 
                        child that terminated normally. \\
                        & \\
                        & \be{WEXITSTATUS(status)} can be called after WIFEXITED() \\
                        & to fetch the actual return status. \\ 
    \hline
    WIFSIGNALED(status) & True if status was returned for a 
                        child that terminated abnormally. \\
                        & \\
                        & \be{WTERMSIG(status)} can be called after
                        WIFSIGNALED() to \\ 
                        & fetch the signal number that caused the termination. \\
                        & \\
                        & \be{WCOREDUMP(status)} returns true if a core file of the \\
                        & terminated process was generates. \\

    \hline
    WIFSTOPPED(status)  & True if status was returned for a 
                        child that is currently stopped.\\
                        &\\
                        &\be{WSTOPSIG(status)} can be called to fetch the signal number \\ 
                        & that caused the child to stop. \\
    \hline
    WIFCONTINUE(status) & True if status was returned for a child that has been continued \\
                        & after a job control stop (XSI extention to POSIX.1; \\
                        & \be{waitpid()} only). \\
    \hline
\end{tabular}
\end{center}

\subsection{About \be{wait()} and \be{waitpid()}}
When a process terminates, either normally or abnormally, the kernel notifies
the parent by sending the \be{SIGCHLD} signal (asynchronously) to the parent.\newline

\textbf{Remember:} If a child is terminated \emph{before} the parent and the
parent is not notified of it's exit status (\be{wait()} or \be{waitpid()} was 
not called), then that child turns into a \textbf{Zombie}.\newline

A process that calls \be{wait()} or \be{waitpid()} can:
\begin{itemize}
    \item Block, if all tis children are still running.
    \item Return immediately with the termination status of a child, if a child has
        terminated and is waiting for its termination status to be feched.
    \item Return immediately with an error, if it doesn't have any child processes.
\end{itemize}

The differences between \be{wait()} and \be{waitpid()} are as follows:
\begin{itemize}
    \item The \be{wait()} function can block the caller until a child process 
        terminates, whereas \be{waitpid()} has an option that prevents it from 
        blocking.
    \item The \be{waitpid()} function doesn't wait for the child that terminates first; 
        \begin{itemize}
            \item In other words, if there is more than one child, \be{wait()} 
                returns on termination of \emph{any} of the children, whereas \be{waitpid()} can 
            choose which child to wait on.
        \end{itemize}
        it has a number of options that control which process it waits for.
\end{itemize}

\clearpage
3 additional features provided by \be{waitpid()}:
\begin{itemize}
    \item Allows us to wait for one particular process, whereas the \be{wait} function 
        returns the status of the first terminated child.
    \item Provides a \emph{nonblocking} version of \be{wait()}.
    \item Provies support for \emph{job control} with the \be{WUNTRACED} 
        and \be{WCONTINUED} options.
\end{itemize}
Additional Info About the Wait Functions
\begin{itemize}
    \item If the process is calling \be{wait()} because it received the \be{SIGCHLD}
        signal, we expect \be{wait()} to return immediately. But if we call it at any 
        random point in time, it can block.
    \item If a child has already terminated and is a zombie, \be{wait()} returns
        immediately with that child's status.
    \item If you don't care about the children's termination status, the \be{*statloc} parameter can be \be{NULL}.
\end{itemize}

\subsection{Other Wait Functions}
\begin{lstlisting}[
    frame = single,
    caption = {Other wait functions mentioned in the book.},
    label = otherwaitfunctions
    ]
    #include <sys/wait.h>

    // Only Solaris provides this I believe
    int waitid(idtype_t idtype, id_t id, siginfo_t *infop, int options);

    #include <sys/types.h>
    #include <sys/wait.h>
    #include <sys/time.h>
    #include <sys/resource.h>

    // These are available in FreBSD, Linux, OSX, Solaris
    pid_t wait3(int *statloc, int options, struct rusage *rusage);
    pid_t wait4(pid_t pid, int *statloc, int options, struct rusage *rusage);

\end{lstlisting}

\section{Race Conditions}
\subsection{Basics}
\begin{itemize}
    \item Occurs when multiple processes are trying to do something with shared data
        and the final outcome depends on the order in which the processes run.
    \item If a process wants to wait for a \emph{child} to terminate, use one of the \textbf{wait} functions.
    \item If a process wants to wait for its \emph{parent} to terminate, a loop of the following form could
        be used (AKA Polling)
        \begin{lstlisting}[frame = single, caption = {Example of Polling}]
    while (getppid() != 1)
        sleep(1);
    \end{lstlisting}
\item The problem with \textbf{polling} is that it wastes CPU time.
\item To avoid race conditoins and to avoid polling, some form of \textbf{signaling} is required
    between multiple processes.
        \begin{itemize}
            \item Various forms of \textbf{interprocess communication (IPC)} can also be used.
        \end{itemize}

\end{itemize}

\section{\be{exec()} Functions}
\subsection{Function Prototypes}
\begin{lstlisting}[
    frame   = single,
    caption = {List of exec functions},
    label   = {execfunctions}
    ]
    #include <unistd.h>

    // 'e' as in 'environment'
    // 'v' as in 'vector'

    /* 'p' for 'filename'. E.g. 'bash'
     * It has a 'p' because it looks up a filename (bash)
     * in your PATH. This way you don't have to specify
     * the WHOLE path. E.g. '/usr/bin/bash'
     */

    // 'pathname' == 'whole path to file'. E.g. '/usr/bin/bash'
    int execl(const char *pathname, const char *arg0, ..., (char *) NULL);

    int execle(const char *pathname     , 
                     const char *arg0   , 
                     ...                , 
                     (char *) NULL      , 
                     char * const env[]);

    int execv(const char *pathname, char * const argv[]);

    int execve(const char *pathname     , 
                     char * const argv[], 
                     char * const envp[]);

    int execlp(const char *filename, const char *arg0, ..., (char *) NULL);

    int execvp(const char *filename, char * const argv[]);

    int execl(const char *pathname, const char *arg0, ..., (char *) NULL);

                                        All 6 return -1 on error, no return on success
\end{lstlisting}
\subsection{About Exec Functions}
\begin{itemize}
    \item When a process calls one of the \be{exec()} functions, that process
        is completely replaced by the new program, and the new program
        starts executing at its \textbf{main} function.
    \item The PID does not change across an \be{exec()}, because a new process 
        is not created;
        \begin{itemize}
            \item \be{exec()} just replaces the current process--its data, heap, 
                and stack segments--with a brand new program from disk.
        \end{itemize}

\end{itemize}
\clearpage
\subsection{Inherited Properties Accross Execs}
The PID does not change after an \be{exec()}, but the new program 
inherits additional properties from the calling process.
\begin{itemize}
    \item PID and PPID
    \item UID and GID
    \item Supplementary GIDs
    \item PGID
    \item SID
    \item Controlling Terminal
    \item Time left until alarm clock
    \item CWD
    \item Root Directory
    \item File mode creation mask
    \item File locks
    \item Process signal mask
    \item Pending signals
    \item Resource limits
    \item Values for \be{tms\_utime}, \be{tms\_stime}, \be{tms\_cutime}, \be{tms\_cstime}
\end{itemize}

\subsection{Other Characteristics Accross Execs}
\begin{itemize}
    \item The default is to leave descriptors open accross the \be{exec()}
        unless we specifically set the close-on-exec flag using \be{fcntl}.
    \item POSIX.1 specifically equires that open directory streams (\be{opendir()})
        be closed accros an \be{exec()}.
    \item Real User ID (UID) an Real Group ID (GID) remain the same accros the 
        \be{exec()}, but the Effective IDs (EID) can change, depending on the 
        status of the Saved Set-User-ID.
\end{itemize}

\section{Changing User IDS (UIDs) and Group IDs (GIDs)}
\subsection{Function Prototypes}
\begin{lstlisting}[
    frame   = single,
    caption = {User/Group ID Function Prototypes},
    label   = uidprototypes
    ]
    #include <unistd.h>

    // UID == Real User ID
    int setuid(uid_t uid);
    int setgid(gid_t gid);

    // EUID == Effective User ID
    int seteuid(uid_t uid);
    int setegid(gid_t gid);
                                                All return: 0 if OK, -1 on error
\end{lstlisting}
In the UNIX System, privileges, such as being able to change the system's notion
of the current date, and access control, such as being able to read or write a particular 
file, are based on user and group IDs. \newline

When our programs need additional privileges or need to gain access to resources that 
they currently aren't allowed to access, they need to change their User or Group ID 
to an ID that has the appropriate privileges 
or access. \newline

Similarly, when our programs need to lower their privileges or prevent acces to certain
resources, they do so by changing either their User ID or Group ID to an ID without
the privilege or ability access to the resource. \newline

\section{Interpreter Files: Nothing Here}
This is pretty easy, just google it real quick.
\section{\be{system()} Function}
\subsection{Function Prototype}
\begin{lstlisting}[
    frame   = single,
    caption = {system() prototype},
    label   = {systemprototype}
    ]
    #include <stdlib.h>

    /* If either fork() fails or waitpid() returns an error other than
     * EINTR, system() returns -1 with errno set to indicate the error.
     *
     * If exec() fails, implying that the shell can't be executed, the return 
     * value is as if the shell had executed exit(127).
     *
     * Otherwise, all 3 functions--fork(), exec(), and waitpid()--succeeded, and
     * the return value from system() is the termination status of the shell, in 
     * the format specified for waitpid().
     */

    int system(const char * command);
\end{lstlisting}
Returns 
\begin{itemize}
    \item If either \be{fork()} fails or \be{waitpid()} returns an error other than 
        \be{EINTR}, \be{system()} returns -1 with \be{errno} set to indicate the error.
\end{itemize}
\subsection{About system()}
The \be{system()} function combines the\be{fork()}, \be{exec()}, and \be{waitpid()} functions
into one.
\subsection{Basic Example}
\begin{lstlisting}[
    frame   = single,
    caption = {A basic use case for system()},
    label   = {systemexample}
    ]
    #include <stdlib.h>

    int main()
    {
        system("echo 'hello there' > hello.txt && cat hello.txt");
    }

                                                                Output:
                                                                hello there
\end{lstlisting}
\subsection{Recreate A Basic system() function}
\begin{lstlisting}[
    frame = single,
    caption = {Simple system() recreation},
    label = {mysystemfunc}
    ]
    #include <sys/wait.h>
    #include <errno.h>
    #include <unistd.h>
    #include <stdlib.h>

    int system(const char * command)
    {
        pid_t pid;
        int   status;

        if (command == NULL)
            return(EXIT_FAILURE);
        
        pid = fork();

        if (pid < 0) 
            status = -1;
        // Child Process
        else if (pid == 0)
        {
            execl("/bin/bash", "random string", "-c", command, (char *) NULL);
            _exit(127);
        }

        /* pid > 0
         *
         * Waits for any child whose process group ID (PGID) 
         * equals the calling process ID.
         */
        else 
        {
            while (waitpid(pid, &status, 0) < 0)
            {
                if (errno != EINTR)
                {
                    status -1;
                    break;
                }
            }
        }
        return status;
    }
\end{lstlisting}

Explanation
\begin{itemize}
    \item Example command: \be{system("ls /usr/bin $|$ grep \string^b $>$ out.txt")}
    \item The shell's \be{-c} option allows us to pass in a command as if we had the terminal open.
    \item The advantage in using \be{system()} instead of \be{fork()} and \be{exec()} directly,
        is that \be{system()} \textbf{does not require error handling and all the required signal handling.}
\end{itemize}


\section{Set-User-ID Programs}
\subsection{Running Set-User-ID Programs As Superuser}
It is a security hold to call \be{system()} from a set-user-ID program and should never be done. \newline

Below is an example of such a program. Imagine running it as a super user. How could this
be dangerous?
\subsection{Security Hole Example}
\begin{lstlisting}[
    frame   = single,
    caption = {Dangerous program!},
    label   = {dangerousprogram}
    ]

    #include <stdlib.h>
    #include <stdio.h>

    int main(int argc, char * argv[])
    {
        int status;

        if (argc < 2)
        {
            printf("Error: command line argument required.");
            exit(EXIT_FAILURE);
        }

        if ((status = system(argv[1])) < 0)
        {
            perror("ERROR");
            exit(EXIT_FAILURE);
        }

        printf("Status: %d\n", status);
        exit(EXIT_SUCCESS);
    }
\end{lstlisting}

Perhaps this will help. What if we were to run the following commands?
\begin{lstlisting}[
    frame = single,
    ]
    $ gcc main.c
    $ su - root
    # ./a.out "rm -rf /"
    
\end{lstlisting}

\textbf{Running this command would remove everything on this person's computer!}\newline

To be safe, if a program is running with special permissions--either set-user-ID or set-group-ID--and
wants to spawn another process, a process should use \be{fork()} and \be{exec()} \textbf{directly}.
This will ensure that permissions change back to normal after the \be{fork()}, before
calling \be{exec()}. \newline

\textbf{Never run \be{system()} from a set-user-ID or a set-group-ID program.}

\section{Process Accounting: Nothing Here}
Boring. Look in the book.

\newpage
\section{User Identification}
\subsection{\be{getlogin()} Prototype}
\begin{lstlisting}[
    frame = single,
    caption = {Get User Login Name},
    belowskip = -0.8 \baselineskip
    ]
    
    // Equivalent to Bash's "$ echo $USER" or "$ echo $LOGNAME"
    #include <unistd.h>

    char * getlogin(void);

                                Returns: char * of login name if OK, NULL on error.
\end{lstlisting}

\section{Process Times}
\subsection{Function Prototype}
\begin{lstlisting}[
    frame = single,
    caption = {times() prototype},
    ]
    #include <sys/times.h>

    clock_t times(struct tms * buf);
                Returns: elapsed wall clock time in clock ticks if OK, -1 on Error
\end{lstlisting}
\subsection{tms Struct}
\begin{lstlisting}[
    frame = single,
    caption = {tms (Times) Struct}
    ]

    /* Note: This struct does not contain wall clock time
     * This is because its return value IS the wall clock time.
     *
     * Also, these times are reported in CLOCK TICKS specified
     * by sysconf(_SC_CLK_TCK);
     * 
     * To get this time into seconds, you must do the following.
     * long clock_ticks = sysconf(_SC_CLK_TCK);
     * long seconds = tms_utime / (double) clock_ticks;
     */

    struct tms
    {
        clock_t tms_utime;  // user CPU time
        clock_t tms_stime;  // system CPU time
        clock_t tms_cutime; // user CPU time of children
        clock_t tms_cstime; // system CPU time of children
    }
\end{lstlisting}
\subsection{Explanation}
3 Types of Process Times:
\begin{itemize}
    \item \textbf{Wall Clock Time}
    \item \textbf{User CPU Time:} Time spent running your application's functions.
    \item \textbf{System CPU Time:} Time spent running OS (Kernel) functions.
\end{itemize}
\subsection{\be{times()} Function And \be{tms} Struct Example Code}
\begin{lstlisting}[
    frame = single,
    caption = {Timing Program Execution}
    ]
    #include <unistd.h>
    #include <stdlib.h>
    #include <stdio.h>
    #include <sys/times.h>
    #include <time.h>

    /* A program that takes in a list of commands and 
     * prints how long they took in various ways.
     */

    static void 
    track_time(clock_t real_time, struct tms * tms_start, struct tms *tms_end);

    int 
    main(int argc, char * argv[])
    {
        struct      tms tms_start, tms_end;
        clock_t     start, end;
        clock_t     real_time;
        int         status;
        char * cmds[] = {"sleep 5", "date"};

        for (int i = 0; i < 2; i++) 
        {
            start     = times(&tms_start);
            system(cmds[i]);
            end       = times(&tms_end);
            real_time = end - start;
            track_time(real_time, &tms_start, &tms_end);
        }

        exit(EXIT_SUCCESS);
    }

    static void 
    track_time(clock_t real_time, struct tms * tms_start, struct tms *tms_end)
    {
        static long clktck = 0;
        clktck = sysconf(_SC_CLK_TCK);

        printf(" real: %.2f seconds\n",  real_time / (double) clktck);
        printf(" user: %.2f seconds\n",
                tms_end->tms_utime - tms_start->tms_utime / (double) clktck);
        printf(" sys:  %.2f seconds\n",
                tms_end->tms_stime - tms_start->tms_stime / (double) clktck);
        printf(" child user: %.2f seconds\n",
                tms_end->tms_cutime - tms_start->tms_cutime / (double) clktck);
        printf(" child sys:  %.2f seconds\n",
                tms_end->tms_cstime  - tms_start->tms_cstime / (double) clktck);
    }
\end{lstlisting}
\end{document}
