\documentclass{article}
\author{Tyler Ryan}
\date{February 15, 2023}
\title{Chapter 11 Notes \\

\textbf{Threads}\\
Advanced Programming in the Unix Environment 3e\\
MCIT 5950}

\usepackage[margin=1.5cm]{geometry}
\usepackage[colorlinks = true, allcolors = {blue}]{hyperref}
\usepackage{parskip}
\usepackage{listings}
\usepackage{fancyhdr}

\pagestyle{fancy}
\fancyfoot{}
\fancyfoot[R]{\thepage}

\begin{document}

% \fancyfoot[LO,CE]{From: K. Grant}
% \fancyfoot[CO,RE]{To: Dean A. Smith}

\maketitle
\tableofcontents
\clearpage

\section{Before Reading}
{\Large Prerequisite Knowledge}

You should know a little about the following subjects:
\begin{itemize}
    \item Generic Types using Void Pointers
    \item Single Linked List
    \item Double Linked List
    \item Queues
    \item Hash Tables (Using Linked Lists)\newline
\end{itemize}

{\Large Installing Missing Man Pages}

Run \textbf{\emph{dnf install man-pages-posix}} to install all \textbf{pthread.h} manual pages. \newline

\section{High Level Overview}
\begin{itemize}    
    \item See how threads of control (i.e. Threads) can be used to perform multiple tasks within the environment of a single process.
    \item All threads within a single process have access to (share) the \textbf{same} process components, such as file descriptors and memory.
        \begin{itemize}
            \item Compare this to this to multiples processes.
        \end{itemize}
    \item Threads and POSIX.1 primitives available to create and destroy them.
    \item Thread synchronization problem.\newline
    \item 3 fundamental synchronization mechanisms:
    \begin{itemize}
        \item Mutexes
        \item Reader-Writer Locks
        \item Conditional Variables
    \end{itemize}
\end{itemize}

\section{Thread Concepts}
\begin{itemize}
    \item With multiple threads (of control), a program can be designed to do more than on thing at a time \textbf{within a single process}, with each thread handling separate tasks.

    \item Benefits of Threads:
\begin{itemize}
    \item Code that deals with asynchronous events can be simplified by assigning a separate thread to handle each event type.
        \begin{itemize}
            \item Each thread can then handle its event using a synchronous programming model.
            \item A synchronous programming model is much simpler than an asynchronous one.
        \end{itemize}
    \item Unlike multiple processes which have to use complex mechanisms provided by the operating system to share memory and file descriptors, \textbf{Threads automatically have access to the same memory address space and file descriptors.}
    \item Unlike processes which implicitly serializes multiple tasks, threads can interleave indepentent tasks by assigning a separate thread per task.
        \begin{itemize}
            \item Two tasks can be interleaved only if they don't depend on the processing performed by each other.
        \end{itemize}
    \item Interactive programs can realize \textbf{improved response time} by using multiple threads to separate the porions of the program that deal with user input and output from the other parts of the program.
\end{itemize}

\item The benefits of a multithreaded programming model can be realized even if your program is running on a uniprocessor.

\item A thread consists of the information necessary to represent an \textbf{execution context} within a process.
    \begin{itemize}
        \item Thread ID
        \item Set of Registers
        \item A Stack
        \item A Scheduling priority and policy
        \item A Signal Mask
        \item An Errno Variable
        \item Thread-Specific Data
    \end{itemize}

\item Everything within a process is sharable among the threads in that process including:
    \begin{itemize}
        \item The text of the executable program
        \item The program's global and heap memory
        \item The Stacks
        \item The File Descriptors
    \end{itemize}
\end{itemize}

\section{Basic Function Prototypes}
\begin{lstlisting}[
    frame = single
    ]
    #include <pthread.h>
    // Data Type
    pthread_t

    // Compare the value of 2 threads
    int pthread_equal(pthread_t tid1, pthread_t tid2);

    // Get current thread's ID
    pthread_t pthread_self(void);

    // Create a thread
    int pthread_create(pthread_t * restrict tidp,
                       const pthread_attr_t * restrict attr,
                       void * (*start_rtn)(void *),
                       void * restrict arg);

    // Returns from a thread
    void pthread_exit(void * rval_ptr);

    // Like wait() for threads
    int pthread_join(pthread_t thread, void **rval_ptr);

    // Like sending pthread_exit() to another thread
    int pthread_cancel(pthread_t tid);

    // Cleanup functions just before threads exit
    void pthread_cleanup_push(void (*rtn)(void *), void * arg);
    // "int execute" is more like "bool execute"
    void pthread_cleanup_pop(int execute);
\end{lstlisting}

\newpage
\subsection{Comparing Process Functions and Thread Functions}
\begin{table}[h!]
        \caption{Process Functions vs Thread Functions}
        \vspace{1.5mm}
    \begin{center}
        \begin{tabular}{|l|l|l|}
            \hline
            \textbf{Process Primitive} & \textbf{Thread Primitive} & \textbf{Description} \\
            \hline
            fork & pthread\_create  & Create a new flow of control \\
            \hline
            exit & pthread\_exit & Exit from an existing flow control \\
            \hline
            waitpid & pthread\_join & Get exit status from flow control \\
            \hline
            atexit & pthread\_cleanup\_push & Register function to be called at exit from flow of control \\
            \hline
            getpid & pthread\_self & Get ID for flow of control \\
            \hline
            abort & pthread\_cancel & Request abnormal termination of flow of control \\
            \hline
            kill & pthread\_kill & Send signals to other flows of control \\
            \hline
        \end{tabular}
    \end{center}
\end{table}

\section{Thread Identification}
\begin{itemize}
    \item Every thread has a Thread ID
    \item These Thread IDs are only unique inside each process.
    \item Each thread shares the same Process ID (PID)
\end{itemize}

\section{Thread Creation}
\begin{itemize}
    \item All programs start as signle threaded processes.

    \item When a thread is created, just like with processes, there is no guarantee which runs first: The newly created thread or the calling thread (program).
\begin{itemize}
    \item Has access to the process address space.
    \item Inherits the calling thread's floating-point environment and signal mask.
    \item However, the set of \emph{pending signals} for the thread is cleared.
\end{itemize}

\item If you need to pass more than one argument to the \textbf{start\_rtn} function,
then you need to store them in a struct and pass the \emph{address} of that struct
to the \textbf{arg} parameter.

\item pthread functions usually return an error code when they fail.
they don't set \textbf{errno} like the other POSIX functions.
\end{itemize}


\subsection{Code: Printing Thread IDs}
\begin{lstlisting}[
    frame = single
    ]

    #include <stdio.h>
    #include <stdlib.h>
    #include <unistd.h>
    #include <pthread.h>
    #include <sys/types.h>
    #include <string.h>

    /* Write a function that prints the current thread id 
     * */

    pthread_t ntid;

    /* Notice how similar the Hex numbers (address ranges) are
     * This means that they are near each other in memory.
     *
     * Running on Linux (Fedora 37), these are the results.
     *
     * PIDs are the same.
     *
     * TID addresses are different,
     *  First 4 positions are the same.
     *  Last 5 positions are the same.
     */
    void 
    print_tid(const char * s)
    {
        pthread_t tid  = pthread_self();
        printf("%s (TID): %u, (HEX) %08x%x, (PID) %u\n", 
                s, 
                (unsigned int) tid,
                (unsigned int) tid,
                (unsigned int) getpid());
    }

    /* When you create a new thread, you must pass in a function.
     * In our case, our function just calls our print_tid() function.
     */
    void *
    thread_fn(void * arg)
    {
        print_tid("new thread: ");

        // Note: In Linux, returning just 0 is allowable.
        return (void *) 0;
    }

    int
    main(void)
    {
        int 	err;

        // This will call a thread that will then print it's TID.
        err = pthread_create(&ntid, NULL, thread_fn, NULL);

        if (err != 0)
        {
            printf("error: can't create thread: %d\n", strerror(err));
            exit(EXIT_FAILURE);
        }

        // Print the main thread's TID
        print_tid("main thread: ");
        sleep(1);
        exit(EXIT_SUCCESS);
    }

\end{lstlisting}
\large{Output}
\begin{verbatim}
    main thread:  (TID): 1975252800, (HEX) 75bbf7401fef9, (PID) 206267015
    new thread:  (TID): 1975248576, (HEX) 75bbe6c01fef9, (PID) 206267015
\end{verbatim}

\clearpage
\section{Thread Termination}
\begin{itemize}
    \item A single thread can exit in 3 ways, stopping its flow of control without terminating the entire process.
        \begin{enumerate}
            \item The thread can return from the start routine.
                \begin{itemize}
                    \item The return value is the thread's exit code
                \end{itemize}
            \item The thread can be canceled by another thread in the \textbf{same process}.
            \item The thread can call \textbf{pthread\_exit()}
        \end{enumerate}


    \item The thread that calls \textbf{pthread\_join()} will block until the specified thread calls \textbf{pthread\_exit()} returns from its start routing or is canceled.
\begin{itemize}
    \item If the thread returns from its routine, \textbf{rval\_ptr} will contain the return code.
    \item If the thread was canceled, the memory location specified by \textbf{rval\_ptr} is set to PTHREAD\_CANCELED.
\end{itemize}

\item By calling \textbf{pthread\_join()}, the joining thread is automatically placed in a detached state so that its resources can be recovered.
\end{itemize}
\subsection{Code: Getting Exit Codes From Threads}
\begin{lstlisting}[
    frame = single
    ]

    #include <stdio.h>
    #include <stdlib.h>
    #include <unistd.h>
    #include <pthread.h>
    #include <string.h>

    /* A simple program that shows how to
     * obtain the exit status and the 
     * return value of a thread using either
     * "return" or "pthread_exit()"
     */

    void *
    thread_func1(void * arg)
    {
        printf("thread 1 returning\n");
        return ((void*)1);
    }

    // See if you can return a char *
    // I'd prefer this one
    void *
    thread_func2(void * arg)
    {
        printf("thread 2 exiting\n");
        // To return a char *
        //pthread_exit("HELLER\n");
        pthread_exit((void*)2);
    }

    int
    main(void)
    {
        pthread_t  thread1, thread2;
        int 	   err;
        void 	 * thread_return;

        // Create thread 1, which returns 1
        err = pthread_create(&thread1, NULL, thread_func1, NULL);
        if (err !=0)
        {
            printf("error: can't create thread 1: %s\n", strerror(err));
            exit(EXIT_FAILURE);
        }

        // Create thread 2, which returns 2 using pthread_exit()
        err = pthread_create(&thread2, NULL, thread_func2, NULL);
        if (err != 0)
        {
            printf("error: can't create thread 2: %s\n", strerror(err));
            exit(EXIT_FAILURE);
        }

        /* Waits for thread 1 and returns the exit status to err, 
         * and your whatever thread_func1() returned to thread_return.
         */
        err = pthread_join(thread1, &thread_return);
        // err == 0
        // thread_return == 1

        if (err !=0)
        {
            printf("error: can't join with thread 1: %s\n", 
                                                strerror(err));
            exit(EXIT_FAILURE);
        }

        printf("thread 1 exit code: %d\n", (int*) thread_return);

        /* Waits for thread 2 and returns the exit status to err, 
         * and your whatever thread_func2() returned to thread_return.
         */
        err = pthread_join(thread2, &thread_return);
        // err == 0
        // thread_return == 2

        if (err !=0)
        {
            printf("error: can't join with thread 2: %s\n", 
                                            strerror(err));
            exit(EXIT_FAILURE);
        }

        printf("thread 2 exit code: %d\n", (int*) thread_return);
        // To return a char *
        //printf("thread 2 exit code: %s\n", (char*) thread_return);

        exit(EXIT_SUCCESS);
    }
\end{lstlisting}
{\Large{Output}}
\begin{verbatim}
    thread 2 exiting
    thread 1 returning
    thread 1 exit code: 1
    thread 2 exit code: 2
\end{verbatim}

\begin{itemize}
    \item Be careful when passing values from thread to thread.
    \item For example, if you create a struct on one thread's stack and try to pass it to another,
    it will return random values as that struct was destroyed when that thread was destroyed.
    \item In essence, when values create within threads die with their threads unless made static or put on the heap.
\end{itemize}
\subsection{Code: Incorrectly Passing Values Between Threads}
\begin{lstlisting}[
    frame = single
    ]
    #include <stdio.h>
    #include <stdlib.h>
    #include <unistd.h>
    #include <pthread.h>


    struct foo { int a, b, c, d; } ;

    void
    print_foo(const char *s, const struct foo *fp)
    {
        printf(s);
        printf(" structure at 0x%x\n", *(unsigned int*)fp);
        printf(" foo.a = %d\n", fp->a);
        printf(" foo.b = %d\n", fp->b);
        printf(" foo.c = %d\n", fp->c);
        printf(" foo.d = %d\n", fp->d);
    }

    void *
    thread_func1(void * arg)
    {
        // Creating a foo struct on this thread's stack
        struct foo foo = {1,2,3,4};
        print_foo("thread 1:\n", &foo);

        /* Because foo was not made static or put on the
         * heap, it will die when we exit this function.
         */
        pthread_exit((void*)&foo);
    }

    void *
    thread_func2(void * arg)
    {
        printf("thread 2: ID is %ld\n", pthread_self());
        pthread_exit((void*)0);
    }

    int
    main(void)
    {
        /* For the sake of simplicity, I'm not handling 
         * any errors
         */
        pthread_t tid1, tid2;
        struct    foo *fp;
        int 	  err;

        // CREATES a foo struct WITHIN this thread
        err = pthread_create(&tid1, NULL, thread_func1, NULL);

        /* Returns a pointer to the foo struct here.
         * However, that thread has already been terminated.
         * If you tried to access the value of fp which points
         * to memory that has been terminated, you will get
         * junk.
         */
        err = pthread_join(tid1, (void*)&fp);

        sleep(1);

        printf("parent starting second thread\n");

        err = pthread_create(&tid2, NULL, thread_func2, NULL);

        sleep(1);
        print_foo("parent:\n", fp);

        exit(EXIT_SUCCESS);
    }
\end{lstlisting}
\clearpage
{\Large Output}
\begin{verbatim}
    thread 1:
     structure at 0x1
     foo.a = 1
     foo.b = 2
     foo.c = 3
     foo.d = 4
    parent starting second thread
    thread 2: ID is 140514740258496
    parent:
     structure at 0x232dc6c0
     foo.a = 590202560
     foo.b = 32716
     foo.c = 0
     foo.d = 0
\end{verbatim}
The parent's output \emph{should} be the same as the first thread, however it isn't because
  its data was destroyed when the first thread exited.
\subsubsection{Code Challenge 1: Returning Multiple Values From Thread}
Try to get this to work. Hint: Use global/heap space.

{\Large Expected Output}
\begin{verbatim}
    thread 1:
     structure at 0x1
     foo.a = 1
     foo.b = 2
     foo.c = 3
     foo.d = 4
    parent starting second thread
    thread 2: ID is 140436664235712
    parent:
     structure at 0x1
     foo.a = 1
     foo.b = 2
     foo.c = 3
     foo.d = 4
\end{verbatim}
See Listing \ref{challengeone} for a \textbf{solution}.

\clearpage
\section{Thread Synchronization}
\subsection{Function Prototypes}
\begin{lstlisting}[
    frame = single
    ]

    #include <pthread.h>

    // -------------------- MUTEX Functions ------------------------
    // Mutex Data Type
    pthread_mutex_t

    // Use for statically-allocated Mutexes
    PTHREAD_MUTEX_INITIALIZER


    int pthread_mutex_int(pthread_mutex_t * restrict mutex
                          const pthread_mutexattr_t * restrict attr);
    int pthread_mutex_destroy(pthread_mutex_t * mutex);

    int pthread_mutex_lock(pthread_mutex_t * mutex);
    int pthread_mutex_trylock(pthread_mutex_t * mutex);
    int pthread_mutex_unlock(pthread_mutex_t * mutex);

                        Return: 0 if OK, error number on failure
    // -------------------------------------------------------------

\end{lstlisting}

\subsection{Mutexes}
A Mutex variable is represented by \textbf{pthread\_mutex\_t}

Use \textbf{PTHREAD\_MUTEX\_INITIALIZER} for statically-allocated mutexes.

\subsubsection{Code: Basic Mutex Implementation}
\begin{lstlisting}[
    frame = single
    ]
    #include <stdio.h>
    #include <stdlib.h>
    #include <unistd.h>
    #include <pthread.h>

    /* Using a signle thread, get this
     * code to work.
     */

    struct foo 
    {
        int 		 f_count;
        pthread_mutex_t  f_lock;
    };

    struct foo *
    foo_alloc(void)
    {
        struct foo * fp;
        fp = malloc(sizeof(struct foo));
        // Keep track of f_lock
        int res = pthread_mutex_init(&fp->f_lock, NULL);

        if (res != 0)
        {
            free(fp);
            return (NULL);
        }

        return fp;
    };

    void
    foo_hold(struct foo * fp)
    {
        // Lock f_lock
        pthread_mutex_lock(&fp->f_lock);
        // Do something
        fp->f_count++;
        // Unlock f_lock
        pthread_mutex_unlock(&fp->f_lock);
    }

    void
    foo_release(struct foo * fp)
    {
        /* "--fp->f_count" is interesting code.
         *
         * What it does is pre-decrement the actual
         * f_count amount by 1.
         *
         * if fp->f_count was 1, then --fp->f_count
         * would == 0.
         *
         * If that is the case, then we can go ahead and
         * do the following:
         * 1. Unlock it
         * 2. Destroy it (make it uninitialized)
         * 3. Free it's memory
         */

        if (--fp->f_count == 0)
        {
            pthread_mutex_unlock(&fp->f_lock);
            pthread_mutex_destroy(&fp->f_lock);
            free(fp);
            fp == NULL;
        }
        /* However, if there are multiple foo objects,
         * then just unlock this particular one.
         */
        else
        {
            pthread_mutex_unlock(&fp->f_lock);
        }
    }

    int
    main(void)
    {
        /* Note: For the sake of simplicity, this example
         * is only showing how to use these functions with
         * 1 thread.
         */
        struct foo * fpointer = foo_alloc();

        // Get 200
        for (int i = 0; i < 200; i++)
        {
            foo_hold(fpointer);
        }

        printf("200: %d\n", fpointner->f_count); // 200

        // Release all but 1
        for (int i = 0; i < 199; i++)
        {
            foo_release(fpointer);
        }

        printf("1: %d\n", fpointer->f_count); // 1
        
        // Release the last one
        foo_release(fpointer);
        // Invalid memory produces junk information
        printf("Destroyed: %d\n", fpointer->f_count);

        exit(EXIT_SUCCESS);
    }
\end{lstlisting}
{\Large Output}
\begin{verbatim}
    200: 200
    1: 1
    Destroyed: 6846
\end{verbatim}

This example shows how to use a mutex with only a single thread.

\clearpage
\subsubsection{Code Challenge 2: Dual Threaded Mutex}
Try running 2 threads. Each thread should increment fpointer by 1 up to 100. So the total should still end up being 200 (i.e. the same output as above).

See Listing \ref{challengetwo} for a \textbf{solution}.
\subsection{Thread Deadlocking}
\begin{itemize}
    \item A thread will deadlock itself if it tries to lock the same mutex twice.
    \item If more than one mutex is used in a program a deadlock can occur if the thread is allowed to hold a mutex and block while trying to lock a second mutext at the same time the other thread holding the second mutex tries to lock the first mutex.
    \item In this case, neither thread can proceed, because each needs a resource that is held by the other, thus the deadlock.
    \item To avoid deadlocks, ensure that when there is a need to acquire two mutexes at the same time, to always lock them in the same order.
\end{itemize}


\subsubsection{Code: Locking Multiple Mutexes}
{\Large Note:}
\begin{itemize}
    \item At this point the textbook gets pretty wild with its examples.
    \item The code in the book wasn't a working example. I've added a few things and made it work using a single thread for simplicity. \newline
\end{itemize}

In this example, deadlocks are avoided by ensuring that when two mutexes need to be acquired at the same time, they are a locked in the same order.

\begin{itemize}
    \item The second mutex protects a hash list that is used to keep track of the \textbf{foo} data structures.
    \item The \textbf{hashlock} mutex protects both the \textbf{fh} hash table and the \textbf{f\_next} hash link field in the \textbf{foo} structure.
    \item The \textbf{f\_lock} mutex in the \textbf{foo} structure protects the acess to the remainder of the \textbf{foo} structure's fields.
\end{itemize}

\begin{lstlisting}[
    frame = single,
    caption = {Multiple Mutex Locking with Hash Table}
    ]

    #include <stdio.h>
    #include <stdlib.h>
    #include <pthread.h>
    /* Please note, the book did not give a 
     * working example of this program.
     */

    // Number of slots
    #define NHASH 12
    /* HASH(fp) basically reduces our hash number to fit 
     * within our NHASH slots
     *
     * Ex. 
     * int hash = 5000;
     * int smaller_hash = HASH(hash); // smaller_hash = 8;
    */
    #define HASH(fp) (((unsigned long)fp) % NHASH)

    // Struct we are trying to protect 
    struct foo 
    {
        int 		f_count;
        pthread_mutex_t f_lock;
        struct foo 	* f_next;
        int 		f_id;
        /* ... more stuff here ... */
    };

    // Functions
    struct foo * foo_alloc(int);
    void   foo_hold(struct foo * fp);
    struct foo * foo_find(int id);
    struct foo * my_find(struct foo * entry);
    void   foo_release(struct foo * fp);
    void   print_ht_values(void);

    // Global variable
    // An array of foo
    struct foo * fh[NHASH];
    // Our global lock/unlock variable
    pthread_mutex_t hashlock = PTHREAD_MUTEX_INITIALIZER;


    int
    main(void)
    {
        /* TODO: Mess around with some stuff you
         * are unsure about here.
         */
        struct foo * a = foo_alloc(150);
        struct foo * b = foo_alloc(200);
        struct foo * c = foo_alloc(300);

        // Note that these both will stay the same thoughout the
        // same execution, but can/will change when reran.
        printf("Actual Value of fp: %ld\n", a);
        printf("Hash value: %ld\n", HASH(a));

        print_ht_values();

        int a_val = foo_find(150)->f_id;
        printf("Just found %d by searching\n", a_val);

        // Uses the hash feature of the hash table to find values
        int my_val = my_find(a)->f_id;
        printf("I just found %d\n", my_val);

        // This doesn't seem to work
        foo_release(a);
        printf("Removed 150\n");
        printf("New Table\n");
        print_ht_values();
    }

    void print_ht_values(void)
    {
        // Loop through the array
        for (int i = 0; i < NHASH; i++)
        {
            long int fp_value = (long int) fh[i];
            int hash_value = HASH(fp_value);
            // Just print the ones that were allocated space
            if (fp_value != 0)
            {
                printf("index %d, fp: 0x%x, val: %d\n", 
                        hash_value, 
                        (unsigned int) fp_value,
                        fh[i]->f_id
                        );
            }
        }
    }

    struct foo * 
    foo_alloc(int id)
    {
        struct foo * fp;
        int    ht_index;

        fp = malloc(sizeof(struct foo));
        // If malloc worked
        if (fp != NULL)
        {
            // Set our count to 1
            fp->f_count = 1;
            
            // If this fails, free fp
            if (pthread_mutex_init(&fp->f_lock, NULL) != 0)
            {
                free(fp);
                return NULL;
            }
            /* fp is just a number we're trying to sqeeze down
             * below our NHASH number.
             * This will create the index where we will store
             * our values
             */
            ht_index = HASH(fp);
            // Lock down the struct so we can mess with it safely
            pthread_mutex_lock(&hashlock);
            // point next node to this index
            fp->f_next = fh[ht_index];
            // Set value of that index to this fp
            fh[ht_index] = fp;
            fp->f_id = id;
            
            // Never allowed to unlock this node?
            pthread_mutex_lock(&fp->f_lock);
            // Unlock struct
            pthread_mutex_unlock(&hashlock);
            /* ... continue initializing ...*/
        }
        return fp;
    }

    void 
    foo_hold(struct foo * fp)
    {
        pthread_mutex_lock(&hashlock);
        fp->f_count++;
        pthread_mutex_unlock(&hashlock);
    }

    // this doesn't utilize the hash feature when finding
    // values in the hash table lol
    // It just loops through everything.
    struct foo * 
    foo_find(int id)
    {
        struct foo * fp;
        int 	   ht_index;

        pthread_mutex_lock(&hashlock);
        // For each index of the array
        for (ht_index = 0; ht_index < NHASH; ht_index++)
        {
            // for each node at each index
            for (fp = fh[ht_index]; fp != NULL; fp = fp->f_next)
            {
                // If the node's id matches our id
                if (fp->f_id == id)
                {
                    // increase fp's memory location by 1?
                    fp->f_count++;
                    pthread_mutex_unlock(&hashlock);
                    return fp;
                }
            }
        }

        // ID couldn't be found
        pthread_mutex_unlock(&hashlock);
        return NULL;
    }

    /* This way uses the hash feature.
     * Notice how I've reduced the nested for loop
     * to just a single for loop.
     *
     * If you were to assert that all keys were unique,
     * then you could create a hash function where you
     * could look up the actual value (instead of the object)
     * like you can with the book's function.
     */
    struct foo * 
    my_find(struct foo * entry)
    {
        struct foo * fp;
        int 	   ht_index;

        pthread_mutex_lock(&hashlock);

        unsigned int hash_code = HASH(entry);
        printf("Hash Code %d\n", hash_code);

        for (fp = fh[hash_code]; fp != NULL; fp = fp->f_next)
        {
            // If the node's id matches our id
            if (fp->f_id == entry->f_id)
            {
                // increase fp's memory location by 1?
                fp->f_count++;
                pthread_mutex_unlock(&hashlock);
                return fp;
            }
        }
        // ID couldn't be found
        pthread_mutex_unlock(&hashlock);
        return NULL;
    }

    // This function DOES utilize the hash feature
    void 
    foo_release(struct foo * fp)
    {
        struct foo  *temp_foo_ptr;
        int 	    ht_index;

        pthread_mutex_lock(&hashlock);

        // If this is the last "instance"
        if (--fp->f_count == 0)
        {
            ht_index 	 = HASH(fp);
            temp_foo_ptr = fh[ht_index];

            // If temp_fp happens to be the first node
            if (temp_foo_ptr == fp)
            {
                // Hashtable[index] = this node
                fh[ht_index] = fp->f_next;
            }
            // else loop through the nodes, until 
            // the correct node is found.
            else
            {
                while (temp_foo_ptr->f_next != fp)
                {
                    temp_foo_ptr = temp_foo_ptr->f_next;
                }
                // At this point, if it's there
                // this will set us there.
                temp_foo_ptr->f_next = fp->f_next;
            }

            // Remove that one from the table
            // 1. Unlock the table
            // 2. Destroy the node's lock
            // 3. Free from memory
            pthread_mutex_unlock(&hashlock);
            pthread_mutex_destroy(&fp->f_lock);
            free(fp);
        }
        // If it wasn't found, unlock table and do nothing.
        else
        {
            pthread_mutex_unlock(&hashlock);
        }
    }
\end{lstlisting}

\clearpage
\subsection{Reader-Writer Locks}
\subsubsection{Function Prototypes}
\begin{lstlisting}[
    frame = single
    ]
    #include <pthread.h>

    // Initialize
    int pthread_rwlock_init(pthread_rwlock_t * restrict rwlock,
                            const pthread_rwlockattr_t * restrict attr);
    // Destroy/Clean Up
    int pthread_rwlock_destroy(pthread_rwlock_t * rwlock);


    // Read Lock
    int pthread_rwlock_rdlock(pthread_rwlock_t * rwlock);
    // Write Lock
    int pthread_rwlock_wrlock(pthread_rwlock_t * rwlock);
    // Unlock
    int pthread_rwlock_unlock(pthread_rwlock_t * rwlock);


    // Try Read
    int pthread_rwlock_tryrdlock(pthread_rwlock_t * rwlock);
    // Try Write
    int pthread_rwlock_trywrlock(pthread_rwlock_t * rwlock);

\end{lstlisting}
\begin{itemize}
    \item Also called \textbf{shared-exclusive locks}.
    \begin{itemize}
        \item Read mode == shared mode.
        \item Write mode == exclusive mode.
    \end{itemize}
\item Reader-writer locks are similar to mutexes, except that they allow for higher degrees of parallelism.
\item With a mutex, the state is either locked or unlocked, and only one thread can lock it at a time.
\item Only one thread at a time can hold a reader-writer lock in write mode, but multiple threads can hold a reader-writer lock in read mode at the same time.
\item Reader-writer locks are well suited for situations in which data structures are read more often than they are modified.


\item When a reader-writer lock is write-locked, all threads attempting to lock it block until it is unlocked.
\item When a reader-writer lok is read-locked, all threads attempting to lock it in read modeare givenaccess, but any threads attempting t olock it in write mode block until all the threads have relinquished their read locks.
    \begin{itemize}
        \item Implementations vary, but reader-writer locks usually block additional readers if a lock is already held in read mode and a thread is blocked trying to acuquire the lock in write mode.
    \end{itemize}
\item When a read-writer lock is held in write mode, the data structure it protects can be modified safely, since only one thread at a time can hold the lock in write mode.
    \begin{itemize}
        \item Prevents constant stream of readers from starving waiting writers.
    \end{itemize}
\end{itemize}

\subsection{Condition Variables}
\subsubsection{Function Prototypes}
\begin{lstlisting}[
    frame = single
    ]
    #include <pthread.h>

    // Data type
    pthread_cond_t

    // For statically-allocated condition variables.
    PTHREAD_COND_INITIALIZER


    int pthread_cond_init(pthread_cond_t * restrict cond,
                          pthread_condattr_t * restrict attr);
    int pthread_cond_destroy(pthread_cond_t * cond);


    int pthread_cond_wait(pthread_cond_t * restrict cond,
                          pthread_mutex_t * restrict mutex);
    int pthread_cond_timedwait(pthread_cond_t * restrict cond,
                               pthread_mutex_t * restrict mutex,
                               const struct timespec * restrict timeout);
    // timespec structure used in ...timedwait()
    struct timespec 
    {
        time_t tv_sec;  // Seconds
        long   tv_nsec; // Nanoseconds
    }


    // Wakes up 1 thread waiting on a condition
    int pthread_cond_signal(pthread_cond_t * cond);
    // Wakes up all threads waiting on a condition
    int pthread_cond_broadcast(pthread_cond_t * cond):
\end{lstlisting}
\begin{itemize}
    \item Condition variables are another synchronization mechanism available to threads.
    \item When used with mutexes, condition variables allow threads to wait in a race-free way for arbitrary conditions to occur.
    \item The condition itself is protected by a mutex.
    \item A thread must first lock the mutext to change the condition state.
    \begin{itemize}
        \item Other threads will not notice the change until they acquire the mutex.
    \end{itemize}
        \begin{itemize}
            \item As the mutext must be locked to be able to evaluate the condition.
        \end{itemize}
\end{itemize}

\clearpage
\section{Code Challenge Solutions}
These are just some problems I came up with. 

They might or might not:
\begin{itemize}
    \item Help
    \item Be correct
    \item Be worth your time
\end{itemize}
\subsection{Challenge 1}
Here I used \textbf{malloc} to solve this problem, but you could've also used \textbf{static}.
\begin{lstlisting}[
    frame = single,
    label = {challengeone},
    caption = {Solution: Returning Multiple Values From A Thread}
    ]

    #include <stdio.h>
    #include <stdlib.h>
    #include <unistd.h>
    #include <pthread.h>
    #include <string.h>

    /* Write a program that returns a struct
     * (i.e. multiple values) from a thread
     * using malloc
     */

    struct foo { int a, b, c, d; } ;

    void
    print_foo(const char *s, const struct foo *fp)
    {
        printf(s);
        printf(" structure at 0x%x\n", *(unsigned int*)fp);
        printf(" foo.a = %d\n", fp->a);
        printf(" foo.b = %d\n", fp->b);
        printf(" foo.c = %d\n", fp->c);
        printf(" foo.d = %d\n", fp->d);
    }

    // Create a function struct within the thread and returns it
    void *
    thread_func1(void * arg)
    {
        // If you don't put foo on the heap, it will die
        // when this thread dies.
        struct foo * foo = (struct foo*) malloc(sizeof(struct foo));
        foo-> a = 1;
        foo-> b = 2;
        foo-> c = 3;
        foo-> d = 4;


        // Proof foo exists within the thread
        print_foo("thread 1:\n", foo);
        pthread_exit((void *)foo);
    }

    void *
    thread_func2(void * arg)
    {
        printf("thread 2: ID is %ld\n", pthread_self());
        pthread_exit((void*)1);
    }


    void print_error(char * msg, int err)
    {
        printf("%s\n", strerror(err));
        exit(EXIT_FAILURE);
    }

    int
    main(void)
    {
        int 	  err;
        struct 	  foo *fp;
        pthread_t tid1, tid2;

        // Creates a foo struct within this thread
        err = pthread_create(&tid1, NULL, thread_func1, NULL);
        if (err != 0) print_error("",err);

        // Returns a pointer to the foo struct here
        err = pthread_join(tid1, (void*)&fp);
        if (err != 0) print_error("",err);

        sleep(1);

        printf("parent starting second thread\n");

        err = pthread_create(&tid2, NULL, thread_func2, NULL);
        if (err != 0) print_error("",err);

        sleep(1);
        print_foo("parent:\n", fp);

        // You must free when you use malloc
        free(fp);

        exit(EXIT_SUCCESS);
    }

\end{lstlisting}

\clearpage
\subsection{Challenge 2}
\begin{lstlisting}[
    frame = single,
    label = {challengetwo},
    caption = {Solution: Dual Threaded Mutex}
    ]
    #include <stdio.h>
    #include <stdlib.h>
    #include <unistd.h>
    #include <pthread.h>

    /* Using multiple threads,
     * get fp->f_count to equal to 200.
     */

    struct foo 
    {
        int 		 f_count;
        pthread_mutex_t  f_lock;
    };

    struct foo *
    foo_alloc(void)
    {
        struct foo * fp;
        fp = malloc(sizeof(struct foo));
        // Keep track of f_lock
        int res = pthread_mutex_init(&fp->f_lock, NULL);

        if (res != 0)
        {
            free(fp);
            return (NULL);
        }

        return fp;
    };

    void
    foo_hold(struct foo * fp)
    {
        // Lock f_lock
        pthread_mutex_lock(&fp->f_lock);
        // Do something
        fp->f_count++;
        // Unlock f_lock
        pthread_mutex_unlock(&fp->f_lock);
    }

    void
    foo_release(struct foo * fp)
    {
        /* "--fp->f_count" is interesting code.
         *
         * What it does is pre-decrement the actual
         * f_count amount by 1.
         *
         * if fp->f_count was 1, then --fp->f_count
         * would == 0
         *
         * If that is the case, then we can go ahead and
         * do the following:
         * 1. Unlock it
         * 2. Destroy it (make it uninitialized)
         * 3. Free it's memory
         */
        if (--fp->f_count == 0)
        {
            pthread_mutex_unlock(&fp->f_lock);
            pthread_mutex_destroy(&fp->f_lock);
            free(fp);
        }

        else 
        {
            pthread_mutex_unlock(&fp->f_lock);
        }
    }

    void 
    empty_foo(struct foo * fp)
    {
        int total = fp->f_count;

        for (int i = 0; i < total; i++)
        {
            foo_release(fp);
        }
    }

    void * 
    thread_func(void * arg)
    {
        for (int i = 0; i < 100; i++)
        {
            foo_hold(arg);
        }
    }

    int
    main(void)
    {
        struct foo * fp = foo_alloc();
        int 	  status;
        pthread_t tid1, tid2;

        // returns 0 upon success. I'm just not catching 
        // these errors for simplicity's sake.
        status = pthread_create(&tid1, NULL, thread_func, fp);
        status = pthread_create(&tid2, NULL, thread_func, fp);

        if (pthread_join(tid1, NULL) != 0)
        {
            printf("error: thread 1\n");
            exit(EXIT_FAILURE);
        }

        // Same thing as above with different syntax
        if (pthread_join(tid2, NULL))
        {
            printf("error: thread 2\n");
            exit(EXIT_FAILURE);
        }

        printf("200: %d\n", fp->f_count);

        // Unlock, Destroy, and Free structure memory
        empty_foo(fp);
        // Now has junk data
        printf("Destroyed: %d\n", fp->f_count);

        exit(EXIT_SUCCESS);
    }

\end{lstlisting}

\end{document}
